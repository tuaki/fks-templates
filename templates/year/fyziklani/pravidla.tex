\expandafter\ifx\csname classoptions\endcsname\relax
\def\classoptions{}
\fi
\documentclass[fykos,\classoptions]{fksgeneric}

\usepackage[fyziklani2]{fkslegacyloader}

\problemsdir{./problems}
\renewcommand\problemnum[2]{\arabic{#2}}
\renewcommand\problemstats{\relax}
\renewcommand\problempoints{5 – 3 – 2 – 1}
\leftheader{MFNáboj}
\rightheader{I. ročník \qquad 13. února 2012}

\begin{document}

\section{Pravidla FYKOSího Fyziklání}

\maketitle

\subsection*{Účast ve hře}
\begin{compactitem}
	\item Aby tým mohl soutěžit, je nutné se přihlásit pomocí webového rozhraní \\
	na \url{http://fyziklani.cz}.
	\item Přihlášením do soutěže se tým zavazuje, že se seznámil s~těmito pravidly
	a~že je bude dodržovat.
	\item Přihlášením do soutěže souhlasíte se zveřejněním výsledků ve formě
	základních údajů (svého jména, příjmení, kategorie, školy a~bodů) ve výsledkové
	listině na internetových stránkách, v~brožurkách a~ročenkách FYKOSu.
	\item Tým se skládá z~1--5 hráčů.
	\item Členové týmu musí být studenty střední, popřípadě základní školy.
	\item Tým nesmí být složen ze studentů z~více než dvou škol.
	\item Maximálně dva týmy můžou obsahovat studenty z~jedné školy. V~případě
	volných míst na soutěži, popřípadě i~za jiných okolností, si organizátoři
	vyhrazují právo udělat tomuto pravidlu výjimku.
	\item Soutěží se ve třech kategoriích, do kterých jsou týmy rozděleny podle
	následujícího algoritmu. Studenti mladší prvního ročníku čtyřletého gymnázia
	a~odpovídajících ročníků víceletých gymnázií si přiřadí koeficient hráče~0,
	studenti prvního ročníku koeficient hráče~1, druhého~2, atd. Koeficient týmu se
	spočte jako průměrná hodnota koeficientů hráčů (koeficienty hráče od
	jednotlivých členů se sečtou a~vydělí počtem členů týmu). Tým se zařadí do
	nejnižší kategorie, která mu vyhovuje:
	\begin{compactitem}
		\item \emph{kategorie A}: koeficient týmu $\leq$ 4,
		\item \emph{kategorie B}: koeficient týmu $\leq$ 3 a~max. dva členové
		týmu mají koeficient hráče 4,
		\item \emph{kategorie C}: koeficient týmu $\leq$ 2,
		žádný člen týmu nemá koeficient hráče 4 a~max. dva členové týmu mají
		koeficient hráče 3.
	\end{compactitem}
	\item Všechny kategorie budou mít stejné zadání úloh.
	\item Pro každou kategorii bude samostatná výsledková listina.
	\item Během soutěže mohou účastníci komunikovat pouze se členy svých týmů nebo
	s~organizátory. Jakákoliv interakce s~učiteli, jinými týmy apod. je přísně
	zakázaná.
	\item Týmy mají povoleno používat jakoukoliv literaturu v~papírové podobě.
	Během soutěže je zakázáno používání internetu. Dále jsou povoleny kalkulačky
	a~psací či rýsovací pomůcky. Kalkulačka nesmí umožňovat přístup k~internetu ani
	jakoukoliv formu komunikace (zařízení typu mobilní telefon, tablet, notebook
	apod. tedy nejsou jako kalkulačky povoleny).
\end{compactitem}

\subsection*{Příjezd na soutěž}
\begin{compactitem}
	\item Týmy jsou povinny se dostavit včas. Organizátoři si vyhrazují právo do
	soutěže nevpustit pozdě příchozí týmy.
	\item Týmy jsou povinny se při příchodu registrovat a~uvést přesné udáje
	o~svých členech (ročníky, školy atd.).
	\item Každý tým dostane obálku se zadáním prvních sedmi úloh. Je zakázáno tuto
	obálku otevřít dříve, než k~tomu dá pokyn vedoucí místnosti, ve které tým
	soutěží.
\end{compactitem}

\clearpage
\subsection*{Systém hry}
\begin{compactitem}
	\item Soutěž trvá 3 hodiny.
	\item Každý tým dostane na začátku soutěže 7 úloh, které se snaží vyřešit.
	\item Pokud si tým myslí, že došel ke správnému řešení, vyšle jednoho zástupce
	k~opravovateli, který mu řekne, zdali je řešení špatně nebo dobře. Zástupce
	musí předložit papírek se zadáním úlohy a~s~uvedeným výsledkem.
	\item Správného opravovatele si zástupce vybere na základě čísla úlohy, kterou
	řeší. Přesný algoritmus určení opravovatele bude vysvětlen před soutěží.
	\item Pokud je řešení špatně, zástupce se vrátí ke svému týmu a~počítá dále.
	\item Pokud je řešení dobře, opravovatel označí papírek se zadáním úlohy počtem
	získaných bodů a~pošle zástupce k~vydavači, od kterého dostane novou úlohu.
	\item Úlohy jsou bodovány podle počtu pokusů potřebných pro vyřešení, a~to
	následovně: jeden pokus~--~5~bodů, dva pokusy~--~3~body, tři
	pokusy~--~2~body a~čtyři a~více pokusů~--~1~bod.
	\item Cílem týmu je získat co nejvíce bodů.
	\item Pokud soutěž bude probíhat pomalu, organizátoři si vyhrazují právo vydat
	všem týmům jednu nebo více nových úloh.
	\item Během soutěže jsou promítány aktuální výsledky všech týmů. Ty budou
	skryty 20~minut před koncem soutěže.
	\item Pokud se během soutěže zjistí, že je závažný problém se zadáním některé
	úlohy, organizátoři si vyhrazují právo tuto úlohu vyřadit ze soutěže bez
	jakékoliv kompenzace týmů za čas strávený jejím řešením.
\end{compactitem}

\subsection*{Ukončení soutěže a~vyhlášení vítězů}
\begin{compactitem}
	\item Konec soutěže je vyhlášen vedoucím místnosti, ve které tým soutěží.
	\item Po vyhlášení konce soutěže již žádný tým nemůže vyslat svého zástupce
	k~opravovatelům. Pokud některý člen týmu stál ve frontě ještě před vyhlášením
	konce, může tam zůstat a~jeho úloha bude opravena, ale již má zakázáno používat
	psací pomůcky.
	\item Pokud o~vítězném týmu (popřípadě druhém a~třetím místě) nerozhodne počet
	bodů, bude rozhodnuto podle následujících kritérií (v~tom pořadí, jak jsou
	uvedeny): vyšší průměrný bodový zisk za úlohu, nižší čas potřebný k~vyřešení
	všech úloh (pouze pokud tým, kterého se to týká, všechny úlohy vyřešil)
	a~náhodný los.
\end{compactitem}

\subsection*{Závěrečná ustanovení}
\begin{compactitem}
	\item Organizátoři si vyhrazují právo na drobné změny pravidel před začátem
	soutěže.
	\item Organizátoři mohou diskvalifikovat tým, který se závažně proviní proti
	pravidlům.
	\item V~případě potíží, které nejsou v~těchto pravidlech specifikovány,
	o~jejich řešení rozhoduje hlavní organizátor nebo vedoucí místnosti.
\end{compactitem}

%\clearpage
%\begin{center}
%	\includegraphics[scale=0.5]{graphics/logo-ondrasovka.pdf}
%\end{center}

\vskip0.4cm
\begin{center}
	\includegraphics[width=0.8\textwidth]{graphics/logo-casopis.pdf}
\end{center}

\end{document}